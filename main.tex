\documentclass[12pt]{article}
\usepackage[utf8]{inputenc}
\usepackage[T1]{fontenc}
\usepackage{color}
\usepackage{url}
\usepackage{amsmath}
\usepackage{amsfonts}
\usepackage{geometry}
\usepackage{graphicx}
\usepackage[unicode]{hyperref}
\usepackage{tocloft}
\usepackage{array}   % Za tabele
\usepackage{lipsum}  % Za generisanje uzorka teksta (nije obavezno)
\usepackage{titlesec}

\geometry{top=1in, bottom=1in, left=1in, right=1in}
\hypersetup{colorlinks,citecolor=green,filecolor=green,linkcolor=blue,urlcolor=blue}

\renewcommand{\contentsname}{Sadržaj}
\renewcommand{\cftsecfont}{\bfseries}
\renewcommand{\cftsecpagefont}{\bfseries}
\setlength{\cftbeforesecskip}{0.5cm}
\setlength{\cftsecindent}{0pt}
\setlength{\cftsubsecindent}{1.5cm}

\begin{document}

\begin{titlepage}

    \newcommand{\HRule}{\rule{\linewidth}{0.4mm}}
    \center
    \textsc{\LARGE Matematički fakultet}\\[5cm]

    \HRule\\[0.4cm]
    {\LARGE\bfseries Odgovori na teorijska ispitna pitanja iz geometrije}\\[0.2cm]
    \HRule\\[2cm]

    \vspace{17\baselineskip}
    \begin{minipage}[t]{0.4\textwidth}
        \begin{flushleft}
            \large
            \textit{Radio}\\
            Lazar Jovanović 34/2023
        \end{flushleft}
    \end{minipage}
    \hspace*{1cm}
    \begin{minipage}[t]{0.4\textwidth}
        \begin{flushright}
            \large
            \textit{Profesor}\\
            dr Srđan Vukmirović
        \end{flushright}
    \end{minipage}

    \vfill\vfill\vfill\vfill
    {\large Beograd, 2024/2025}
    \vfill

\end{titlepage}

\tableofcontents
\newpage

\section{Konvencija}

\newpage

\section{Pitanja}
\subsection{Vektori i osnovne operacije sa vektorima}
\textit{Sabiranje i množenje skalarom, nula vektor, suprotan vektor,
    kolinearni i komplanarni vektori, jedinični vektor, Dokaz T1.1
    (tvrđenja S1-S4).}\\
\textbf{Odgovor:}\\
Definicija vektora: Klasa ekvivalencije usmerenih duži koje
imaju isti pravac smer i intenzitet.\\
Sabiranje vektora:
$\overrightarrow{AB}+\overrightarrow{BC}=\overrightarrow{AC}$.\\
Množenje vektora skalarom: Vektor $\alpha\cdot\overrightarrow{v}$
ima isti pravac kao $\overrightarrow{v}$, ima intenzitet
$|\overrightarrow{v}|\cdot|\alpha|$ i ima isti smer ako je $\alpha>0$,
a različit smer ako je $\alpha<0$. Specijalno, ako je $\alpha=0$, svaki
vektor $\overrightarrow{v}$ postaje nula vektor.\\
Nula vektor: Vektor $\overrightarrow{v}$ je nula vektor (zapisujemo
$\overrightarrow{0}$) ako je njegov intenzitet jednak nuli. Nula vektor
nema pravac i smer.\\
Suprotni vektor: Suprotni vektor vektora $\overrightarrow{v}$
(zapisujemo sa $-\overrightarrow{v}$) ima isti intenzitet i pravac kao
vektor $\overrightarrow{v}$, a suprotan smer. Takođe, suprotan vektor
vektora $\overrightarrow{AB}$ možemo da zapišemo kao $\overrightarrow{BA}$\\
Kolinearni vektori: Vektori su kolinearni ako pripadaju istoj pravoj.\\
Komplanarni vektori: Vektori su komplanarni ako pripadaju istoj ravni.\\
Jedinični vektor: Jedinični vektor je vektor koji je dužine $1$. Od
svakog ne-nula vektora možemo da dobijemo jedinični vektor normalizacijom.\\
Normalizacija vektora: Normalizacijom se od vektora
$\overrightarrow{v}$ dobija vektor $\frac{1}{|\overrightarrow{v}|}\cdot\overrightarrow{v}$.\\
T1.1: Neka $\overrightarrow{u}$, $\overrightarrow{v}$, $\overrightarrow{w}\in\mathbb{V}$. Tada važe
sledeća tvrđenja:\\
\textbf{(S1)}
$(\overrightarrow{u}+\overrightarrow{v})+\overrightarrow{w}=\overrightarrow{u}+(\overrightarrow{v}+\overrightarrow{w})$\\
Dokaz:
$(\overrightarrow{AB}+\overrightarrow{BC})+\overrightarrow{CD}=\overrightarrow{AC}+\overrightarrow{CD}=\overrightarrow{AD}
    =\overrightarrow{AB}+\overrightarrow{BD}=\overrightarrow{AB}+(\overrightarrow{BC}+\overrightarrow{CD})$\\
\textbf{(S2)}
$\overrightarrow{u}+\overrightarrow{0}=\overrightarrow{u}=\overrightarrow{0}+\overrightarrow{u}$\\
Dokaz:
$\overrightarrow{AB}+\overrightarrow{BB}=\overrightarrow{AB}=\overrightarrow{AA}+\overrightarrow{AB}$\\
\textbf{(S3)}
$\overrightarrow{u}+\overrightarrow{-u}=\overrightarrow{0}$\\
Dokaz:
$\overrightarrow{AB}+\overrightarrow{BA}=\overrightarrow{AA}$\\
\textbf{(S4)}
$\overrightarrow{u}+\overrightarrow{v}=\overrightarrow{v}+\overrightarrow{u}$\\
Dokaz:
$\overrightarrow{AB}+\overrightarrow{BC}=\overrightarrow{AC}=-\overrightarrow{CA}=-\overrightarrow{CB}-\overrightarrow{BA}=\overrightarrow{BC}+\overrightarrow{AB}$

\par

\subsection{Linearna zavisnost i nezavisnost vektora}
\textit{Definicija, primeri, T1.3 (dokaz), T1.4.}\\
\textbf{Odgovor:}\\
Definicija: Vektori $\overrightarrow{v_1}$, ..., $\overrightarrow{v_n}$ su linearno nezavisni ako važi:
$$\alpha_1\cdot\overrightarrow{v_1}+...+\alpha_n\cdot\overrightarrow{v_n}=0 \iff \alpha_1=0, ..., \alpha_n=0$$
T1.3: U ravni postoje dva linearno nezavisna vektora, a svaka
tri su zavisna.\\
U ravni postoje 3 nekolinearne tačke O, A, B. Tada su
vektori $\overrightarrow{OA}$ i $\overrightarrow{OB}$
linearno nezavisni.\\
Posmatrajmo $\overrightarrow{u}$, $\overrightarrow{v}$ i
$\overrightarrow{w}$ koji pripadaju istoj ravni. Ako su
$\overrightarrow{u}$ i $\overrightarrow{v}$ linearno zavisni
tada $\overrightarrow{u}=\alpha\cdot\overrightarrow{v}$ i
$1\cdot\overrightarrow{u}-\alpha\cdot\overrightarrow{v}+0\cdot\overrightarrow{w}=0$.
Dakle i $\overrightarrow{u}$, $\overrightarrow{v}$ i $\overrightarrow{w}$ su linearno
zavisni.\\
Pretpostavimo da su $\overrightarrow{u}$ i $\overrightarrow{v}$ linearno nezavisni.
Posmatrajmo tačke u ravni O, A, B, C takve da $\overrightarrow{OA}=\overrightarrow{u}$,
$\overrightarrow{OB}=\overrightarrow{v}$, $\overrightarrow{OC}=\overrightarrow{w}$.
Obeležimo X, Y na OA, OB tako da je OXCY paralelogram. Važi
$\overrightarrow{OX}=\alpha\overrightarrow{OA}$ i $\overrightarrow{OY}=\beta\overrightarrow{OB}$.
Dakle: $$\overrightarrow{OX}+\overrightarrow{OY}=\overrightarrow{OC}=\alpha\overrightarrow{OA}+\beta\overrightarrow{OB}\implies \alpha\overrightarrow{OA}+\beta\overrightarrow{OB}-\overrightarrow{OC}=\overrightarrow{0}$$
što je kraj dokaza.\\
T1.4: U prostoru postoje tri linearno nezavisna vektora, a
svaka 4 su linearno zavisni.

\par

\subsection{Koordinate vektora i tačaka}
\textit{Baza i dimenzija vektorskog prostora, koordinate vektora
    u bazi, primer, reper Oe, koordinate tačke u reperu,
    primer, koordinate vektora
    $\overrightarrow{AB}$ preko koordinata tačaka $A$, $B$ (dokaz).
}\\
\textbf{Odgovor:}\\
Definicija baze: Baza je maksimalan skup linearno nezavisnih vektora.\\
Definicija dimenzije: Dimenzija je broj vektora u bazi.\\
Ako je baza vektorskog prostora $e=(\overrightarrow{e_1}, \overrightarrow{e_2})$,
tada se svaki vektor može zapisati kao $\overrightarrow{v}=\alpha_1\cdot\overrightarrow{e_1}+\alpha_2\cdot\overrightarrow{e_2}$
i kažemo da su njegove koordinate $(\alpha_1, \alpha_2)$. Njih zapisujemo
sa $[\overrightarrow{v}]_e=[\alpha_1, \alpha_2]^T$.\\
Ako fiksiramo bazu $e$ i tačku $O$, $Oe$ ćemo zvati reperom ili
koodinatnim sistemom.\\
Koordinate tačke $X$ u $Oe$ su: $[X]_{Oe}:=[\overrightarrow{OX}]_e$.\\
Izraziti koordinate $[\overrightarrow{AB}]$ preko koordinata tačaka $A$ i $B$
$$[\overrightarrow{AB}]_e=[\overrightarrow{AO}+\overrightarrow{OB}]_e=[\overrightarrow{AO}]_e+[\overrightarrow{OB}]_e=-[\overrightarrow{OA}]_e+[\overrightarrow{OB}]_e=[B]_{Oe}-[A]_{Oe}$$

\par

\subsection{Skalarni proizvod}
\textit{Definicija, osobine, pojam ON baze, formula za skalarni proizvod u ON
    bazi (dokaz), računanje dužina i uglova skalarnim proizvodom,
    primeri.}\\
\textbf{Odgovor:}\\
Definicija skalarnog proizvoda: Za vektore $\overrightarrow{u}, \overrightarrow{v}\in\mathbb{V}$, $\overrightarrow{u}\cdot\overrightarrow{v}=|\overrightarrow{u}|\cdot\overrightarrow{v}\cdot\cos(\angle{\overrightarrow{u}\overrightarrow{v}})$\\
Osobine:\\
$\overrightarrow{u}\cdot\overrightarrow{v}=\overrightarrow{v}\cdot\overrightarrow{u}$\hspace*{1cm}(simetričnost)\\
$\overrightarrow{u}\cdot(\alpha\cdot\overrightarrow{v}+\beta\cdot\overrightarrow{w})=\alpha\cdot(\overrightarrow{u}\cdot\overrightarrow{v})+\beta\cdot(\overrightarrow{u}\cdot\overrightarrow{w}$)\hspace*{1cm}(linearnost)\\
$\overrightarrow{u}\cdot\overrightarrow{u}=|\overrightarrow{u}|^2\geq0$\hspace*{1cm}(pozitivnost)\\
$\overrightarrow{u}\cdot\overrightarrow{u}=0\iff\overrightarrow{u}=0$\hspace*{1cm}(nedegenerisanost)\\
    Definicija ON baze: ON baza je svaka baza $e=(\overrightarrow{e_1}, ..., \overrightarrow{e_n})$ za koju važi
    da je svaki vektor jedinični i svaka dva su međusobno ortogonalna. Važi:
$\overrightarrow{e_i}\cdot\overrightarrow{e_j} = \begin{cases}
    0 & i=j     \\
    1 & i\neq j
\end{cases}$\\
    Formula u ON bazi:
    Neka je $\overrightarrow{v}=x_1\cdot\overrightarrow{e_1}+x_2\cdot\overrightarrow{e_2}$ i
$\overrightarrow{u}=y_1\cdot\overrightarrow{e_1}+y_2\cdot\overrightarrow{e_2}$.
    Tada:
    \begin{align*}
        \overrightarrow{v}\cdot\overrightarrow{u} & = (x_1\cdot\overrightarrow{v}+x_2\cdot\overrightarrow{v})\cdot(y_1\cdot\overrightarrow{u}+y_2\cdot\overrightarrow{u})                                                                                                                                                 \\
                                                  & = x_1\cdot y_1\cdot \overrightarrow{e_1}\cdot \overrightarrow{e_1}+x_2\cdot y_1\cdot \overrightarrow{e_1}\cdot \overrightarrow{e_2}+x_1\cdot y_2\cdot \overrightarrow{e_1}\cdot \overrightarrow{e_2}+x_2\cdot y_2\cdot \overrightarrow{e_2}\cdot \overrightarrow{e_2} \\
                                                  & = x_1\cdot y_1+x_2\cdot y_2
    \end{align*}
    Računanje dužine: $|\overrightarrow{v}|=\sqrt{|\overrightarrow{v}|^2}=\sqrt{\overrightarrow{v}\cdot\overrightarrow{v}}$\\
    Računanje ugla: Ako sa $\phi$ označimo ugao između dva vektora tada:\\
$\phi=\arccos(\cos(\phi))=\arccos(\frac{\overrightarrow{u}\cdot\overrightarrow{v}}{|\overrightarrow{u}|\cdot|\overrightarrow{v}|})$\\
    Skalarni proizvod $\overrightarrow{u}\cdot\overrightarrow{v}$ možemo da zapišemo i kao $\overrightarrow{u}^T\cdot\overrightarrow{v}$
    \par
    \subsection{Vektorski proizvod}
    \textit{Orijentacija u ravni, orijentacija u prostoru, definicija vektorskog proizvoda, osobine,
        vektorski proizvod baznih vektora ON+ baze, formula za vektorski
        proizvod u ON+ bazi (dokaz), matrica vektorskog množenja (dokaz),
        računanje površine i orijentacije trougla vektorskim proizvodom
        (dokaz), kolinearnost tačaka, da li tačka pripada trouglu (dokaz),
        primeri.
    }\\
    \textbf{Odgovor:}\\
    Orijentacija u ravni: pozitivna ako je obrnuta od smera kazaljke na satu, inače negativna.
    Orijentacija u prostoru: koristimo pravilo desne ruke.\\
    Definicija vektorskog proizvoda: Intenzitet vektorskog proizvoda $|\overrightarrow{u}\times\overrightarrow{v}|=|\overrightarrow{u}|\cdot|\overrightarrow{v}|\cdot\sin(\angle
\overrightarrow{u}\overrightarrow{v})$. Pravac vektorskog proizvoda je normalan na ravan koju obrazuju vektori $\overrightarrow{u}$ i $\overrightarrow{v}$. Smer je takav
    da je baza $(\overrightarrow{u}, \overrightarrow{v}, \overrightarrow{u}\times\overrightarrow{v})$ pozitivno orijentisana.\\
    Osobine:\\
$\overrightarrow{u}\times\overrightarrow{v}=-\overrightarrow{v}\times\overrightarrow{u}$\hspace*{1cm}(antisimetričnost)\\
$(\alpha\cdot\overrightarrow{u}+\beta\cdot\overrightarrow{v})\times\overrightarrow{w}=\alpha\cdot(\overrightarrow{u}\times\overrightarrow{w})+\beta\cdot(\overrightarrow{v}\times\overrightarrow{w})$\hspace*{1cm}(linearnost)
    \\[0.5cm]
    \noindent
    \begin{minipage}{0.65\textwidth}
        \begin{flushleft}
            Vektorski proizvod baznih vektora u ON+ bazi vidimo iz tabele desno.
            Takođe, na osnovu tabele možemo da napravimo krug sa kog možemo da čitamo rezultate.
            Vektori $\overrightarrow{e_1}$, $\overrightarrow{e_2}$ i $\overrightarrow{e_3}$
            su postavljeni u krug i postoji strelica sa $\overrightarrow{e_1}$ na $\overrightarrow{e_2}$,
            sa $\overrightarrow{e_2}$ na $\overrightarrow{e_3}$ i sa $\overrightarrow{e_3}$ na $\overrightarrow{e_1}$.
            Tada, na primer, ako tražimo vektorski proizvod $\overrightarrow{e_2}$ i $\overrightarrow{e_1}$ vidimo da strelica
            pokazuje sa $\overrightarrow{e_1}$ na $\overrightarrow{e_2}$ pa je znak negativan, a jedini neiskorišćen vektor je
            $\overrightarrow{e_3}$ pa je rešenje $-\overrightarrow{e_3}$.
        \end{flushleft}
    \end{minipage}
    \hfill
    \begin{minipage}{0.3\textwidth}
        \centering
        \begin{tabular}{|c|c|c|c|}
            \hline
            $\times$               & $\overrightarrow{e_1}$  & $\overrightarrow{e_2}$  & $\overrightarrow{e_3}$  \\
            \hline
            $\overrightarrow{e_1}$ & $\overrightarrow{0}$    & $-\overrightarrow{e_3}$ & $\overrightarrow{e_2}$  \\
            \hline
            $\overrightarrow{e_2}$ & $\overrightarrow{e_3}$  & $\overrightarrow{0}$    & $-\overrightarrow{e_1}$ \\
            \hline
            $\overrightarrow{e_3}$ & $-\overrightarrow{e_2}$ & $\overrightarrow{e_1}$  & $\overrightarrow{0}$    \\
            \hline
        \end{tabular}
    \end{minipage}
    Formula za vektorski proizvod u ON+ bazi: Neka je $\overrightarrow{v}=x_1\cdot\overrightarrow{e_1}+x_2\cdot\overrightarrow{e_2}+x_3\cdot\overrightarrow{e_3}$ i
$\overrightarrow{u}=y_1\cdot\overrightarrow{e_1}+y_2\cdot\overrightarrow{e_2}+y_3\cdot\overrightarrow{e_3}$. Tada:
    \begin{align*}
        \overrightarrow{v}\times\overrightarrow{u} & = (x_1\cdot\overrightarrow{e_1}+x_2\cdot\overrightarrow{e_2}+x_3\cdot\overrightarrow{e_3})\times(y_1\cdot\overrightarrow{e_1}+y_2\cdot\overrightarrow{e_2}+y_3\cdot\overrightarrow{e_3})                \\
                                                   & = x_1\cdot y_1\cdot \overrightarrow{e_1}\times \overrightarrow{e_1}+x_1\cdot y_2\cdot \overrightarrow{e_1}\times \overrightarrow{e_2}+x_1\cdot y_3\cdot \overrightarrow{e_1}\times \overrightarrow{e_3} \\
                                                   & + x_2\cdot y_1\cdot \overrightarrow{e_2}\times \overrightarrow{e_1}+x_2\cdot y_2\cdot \overrightarrow{e_2}\times \overrightarrow{e_2}+x_2\cdot y_3\cdot \overrightarrow{e_2}\times \overrightarrow{e_3} \\
                                                   & + x_3\cdot y_1\cdot \overrightarrow{e_3}\times \overrightarrow{e_1}+x_3\cdot y_2\cdot \overrightarrow{e_3}\times \overrightarrow{e_2}+x_3\cdot y_3\cdot \overrightarrow{e_3}\times \overrightarrow{e_3} \\
                                                   & = x_1\cdot y_1\cdot \overrightarrow{0} + x_1\cdot y_2\cdot \overrightarrow{e_3}-x_1\cdot y_3\cdot \overrightarrow{e_2}                                                                                  \\
                                                   & - x_2\cdot y_1\cdot \overrightarrow{e_3}+x_2\cdot y_2\cdot \overrightarrow{0}+x_2\cdot y_3\cdot \overrightarrow{e_1}                                                                                    \\
                                                   & + x_3\cdot y_1\cdot \overrightarrow{e_2}-x_3\cdot y_2\cdot \overrightarrow{e_1}+x_3\cdot y_3\cdot \overrightarrow{0}                                                                                    \\
                                                   & = \begin{vmatrix}
                                                           \overrightarrow{e_1} & \overrightarrow{e_2} & \overrightarrow{e_3} \\
                                                           x_1                  & x_2                  & x_3                  \\
                                                           y_1                  & y_2                  & y_3
                                                       \end{vmatrix}
    \end{align*}
    Matrica vektorskog množenja $\overrightarrow{v}_\times$ je matrica za fiksirani vektor $\overrightarrow{v}$.
    Kada se ona pomnoži nekim vektorom $\overrightarrow{u}$ dobija se $\overrightarrow{v}\times\overrightarrow{u}$.
    Izvodi se izračunavanjem $\overrightarrow{v}\times\overrightarrow{e_1}$ (prva kolona), $\overrightarrow{v}\times\overrightarrow{e_2}$ (druga kolona), $\overrightarrow{v}\times\overrightarrow{e_3}$ (treća kolona).
    Neka je $\overrightarrow{v}=x_1\cdot\overrightarrow{e_1}+x_2\cdot\overrightarrow{e_2}+x_3\cdot\overrightarrow{e_3}$.\\
    Tada  $\overrightarrow{v}_\times= \begin{vmatrix}
    0    & x_3  & -x_2 \\
    -x_3 & 0    & x_1  \\
    x_2  & -x_1 & 0
\end{vmatrix}$.\\
    Površina trougla $ABC$:
    $$P_{ABC}=\frac{1}{2}\cdot a\cdot h_a=\frac{1}{2}\cdot |\overrightarrow{BC}|\cdot|\overrightarrow{BA}|\sin(\angle\overrightarrow{BA}\overrightarrow{BC})=\frac{1}{2}\cdot |\overrightarrow{BC}\times\overrightarrow{BC}|$$
    Trougao $ABC$ u reperu $Oe$ je zadat sa $[A]_{Oe}=(x_1,x_2)^T$, $[B]_{Oe}=(y_1,y_2)^T$ i $[C]_{Oe}=(z_1,z_2)^T$.
    Njegove koordinate možemo da proširimo sa $[A]_{Oe}=(x_1,x_2,0)^T$, $[B]_{Oe}=(y_1,y_2,0)^T$ i $[C]_{Oe}=(z_1,z_2,0)^T$.
    Kažemo da je trougao $ABC$ pozitivno orijentisan ako je
$\overrightarrow{AB}\times\overrightarrow{AC}$ istog pravca i smera kao $\overrightarrow{e_3}$.\\
    Tri tačke su kolinearne ako je "površina trougla" koji obrazuju 0.
    Dakle $\frac{1}{2}\cdot |\overrightarrow{BC}\times\overrightarrow{BC}|=0$.\\
    Tačka $P$ pripada trouglu $ABC$ ako su trouglovi $ABP$, $BCP$ i $CAP$ iste orijentacije.
    \par
    \subsection{Mešoviti proizvod}
    Definicija, osobine, formula za mešoviti proizvod u ON+
    bazi (dokaz), zapremina paralelopipeda - T1.7 (dokaz),
    zapremina tetraedra, orijentacija baze prostora, nezavisnost
    tri vektora prostora, primeri.\\
    \textbf{Odgovor:}\\
    Definicija: $[\overrightarrow{u},\overrightarrow{v},\overrightarrow{w}]=(\overrightarrow{u}\times\overrightarrow{v})\cdot\overrightarrow{w}$\\
    Osobine:\\
$[\overrightarrow{u},\overrightarrow{v},\overrightarrow{w}]=-[\overrightarrow{v},\overrightarrow{u},\overrightarrow{w}]$\hspace*{1cm}(antisimetričnost)\\
$[\overrightarrow{u},\overrightarrow{v},\overrightarrow{w}]=[\overrightarrow{v},\overrightarrow{w},\overrightarrow{u}]$\hspace*{1cm}(cikličnost)\\
$[\alpha\cdot\overrightarrow{u}+\beta\cdot\overrightarrow{v},\overrightarrow{w},\overrightarrow{z}]=\alpha\cdot[\overrightarrow{u},\overrightarrow{w},\overrightarrow{z}]+\beta\cdot[\overrightarrow{v},\overrightarrow{w},\overrightarrow{z}]$\hspace*{1cm}(linearnost)\\
    Formula za mešoviti proizvod u ON+ bazi:\\
$[\overrightarrow{u},\overrightarrow{v},\overrightarrow{w}]=(\overrightarrow{u}\times\overrightarrow{v})\cdot\overrightarrow{w}= \begin{vmatrix}
    \overrightarrow{e_1}   & \overrightarrow{e_2}   & \overrightarrow{e_3}   \\
    x_{\overrightarrow{u}} & y_{\overrightarrow{u}} & z_{\overrightarrow{u}} \\
    x_{\overrightarrow{v}} & y_{\overrightarrow{v}} & z_{\overrightarrow{v}}
\end{vmatrix}\cdot\overrightarrow{w}=\text{*sredi se*}= \begin{vmatrix}
    x_{\overrightarrow{u}} & y_{\overrightarrow{u}} & z_{\overrightarrow{u}} \\
    x_{\overrightarrow{v}} & y_{\overrightarrow{v}} & z_{\overrightarrow{v}} \\
    x_{\overrightarrow{w}} & y_{\overrightarrow{w}} & z_{\overrightarrow{w}}
\end{vmatrix}$\\[0.3cm]
    T1.7: Zapremina paralelopipeda jednaka je mešovitom proizvodu tri vektora.\\
    Dokaz:\\
    Neka je dat paralelopiped $ABCDA_1B_1C_1D_1$ i neka su $\overrightarrow{AB}=\overrightarrow{u}$, $\overrightarrow{AD}=\overrightarrow{v}$ i $\overrightarrow{AA_1}=\overrightarrow{w}$.
    Tada je zapremina $V=P_{ABCD}\cdot h=|\overrightarrow{u}\times\overrightarrow{v}|\cdot h=||\overrightarrow{u}\times\overrightarrow{v}|\cdot \overrightarrow{w}|\cos(\angle \overrightarrow{u}\times\overrightarrow{v},\overrightarrow{w})|=|\overrightarrow{u}\times\overrightarrow{v}\cdot \overrightarrow{w}|=|[\overrightarrow{u},\overrightarrow{v},\overrightarrow{w}]|$.\\
    Zapremina tetraedra je $\frac{1}{6}\cdot |[\overrightarrow{u},\overrightarrow{v},\overrightarrow{w}]|$.\\
    Dokaz:\\
    Posmatrajmo paralelopiped $ABCDA_1B_1C_1D_1$ i u njemu trostranu prizmu $ABCA_1B_1C_1$ sa zapreminom
$\frac{1}{2}\cdot|[\overrightarrow{u},\overrightarrow{v},\overrightarrow{w}]|$. Možemo da izdvojimo tetraedre $ABCC_1$, $AA_1B_1C_1$ i $ABC_1D_1$.
    Tetraedri $ABCC_1$ i $AA_1B_1C_1$ imaju istu površinu baze $ABC$ i $A_1B_1C_1$ i istu visinu $CC_1$ i $AA_1$.
    Tetraedri $ABCC_1$ i $AB_1C_1B$ imaju istu površinu baze $BCC_1$ i $C_1B_1B$ i istu visinu $AB$ i $AB$.
    Zapremina tetraedra je dakle jedna šestina zapremine paralelopipeda.\\
    Orijentacija baze prostora:\\
    Ako je $[\overrightarrow{e_1},\overrightarrow{e_2},\overrightarrow{e_3}]>0$ onda je orijentacija pozitivna.\\
    Tri vektora su nezavisna ako je njihova zapremina veća od nule.
    \par
    \subsection{Težište, centar mase i baricentričke koordinate}
    Zakon poluge, rešavanje zakona poluge centrom mase dve tačke, centar
    mase tri tačke, ”rešavanje trougla” centrom mase tri tačke, centar
    mase n tačaka, težište n tačaka, dokaz da definicija težišta trougla
    ne zavisi od tačke $O$ - vežbe, primeri, baricentričke koordinate.
    \\
    \textbf{Odgovor:}\\
    Zakon poluge: $|\overrightarrow{AT}|:|\overrightarrow{TB}|=m_B:m_A$.\\
    Centar mase tačaka $A(m_A)$ i $B(m_B)$: $\overrightarrow{TA}\cdot m_A+\overrightarrow{TB}\cdot m_B=\overrightarrow{0}$.\\
    Centar mase tri tačke: $\overrightarrow{TA}\cdot m_A+\overrightarrow{TB}\cdot m_B+\overrightarrow{TC}\cdot m_C=\overrightarrow{0}$.\\
    Centar mase n tačaka: $\overrightarrow{TA_1}\cdot m_{A_1}+...+\overrightarrow{TA_n}\cdot m_{A_n}=\overrightarrow{0}$.\\
    Težište $n$ tačaka: $\frac{1}{m_{A_1}+...+m_{A_n}}\cdot(\overrightarrow{OA_1}\cdot m_{A_1}+...+\overrightarrow{OA_n}\cdot m_{A_n})=\overrightarrow{OT}$\\
    Težište $n$ tačaka odgovara centru masa.\\
    Dokaz da definicija težišta trougla ne zavisi od tačke $O$:\\
    Neka je $m_{A}+m_{B}+m_{C}=M$. Tada:
    \begin{align*}
        \overrightarrow{OT}                     & =\frac{1}{M}\cdot(\overrightarrow{OA}\cdot m_{A}+\overrightarrow{OB}\cdot m_{B}+\overrightarrow{OC}\cdot m_{C})                                                                   \\
        \overrightarrow{OS}+\overrightarrow{ST} & =\frac{1}{M}\cdot((\overrightarrow{OS}+\overrightarrow{SA})\cdot m_{A}+(\overrightarrow{OS}+\overrightarrow{SB})\cdot m_{B}+(\overrightarrow{OS}+\overrightarrow{SC})\cdot m_{C}) \\
        \overrightarrow{OS}+\overrightarrow{ST} & =\frac{1}{M}\cdot(\overrightarrow{SA}\cdot m_{A}+\overrightarrow{SB}\cdot m_{B}+\overrightarrow{SC}\cdot m_{C})+\overrightarrow{OS}                                               \\
        \overrightarrow{ST}                     & =\frac{1}{M}\cdot(\overrightarrow{SA}\cdot m_{A}+\overrightarrow{SB}\cdot m_{B}+\overrightarrow{SC}\cdot m_{C})
    \end{align*}\\
    U opštem slučaju za $n$ tačaka $A_1,...,A_n$ čije su mase redom $m(A_1),...,m(A_n)$, baricentričke koordinate će biti
$(\frac{m(A_1)}{m(A_1)+...+m(A_n)},...,\frac{m(A_n)}{m(A_1)+...+m(A_n)})$. Primetimo da važi da je zbir svih baricentričkih
    koordinata jednak $1$.

    \par
    \subsection{Transformacije koordinata vektora i tačaka}
    Matrica prelaska, veza koordinata vektora u različitim bazama,
    primer, transformacija koordinata tačaka - formule (1.23, 1.24)
    izvodjenje, primer.\\
    \textbf{Odgovor:}\\
    Matrica prelaska: matrica čije su kolone koordinata vektora nove baze u staroj.\\
    Transformacije koordinata tačaka:\\
    Pretpostavimo da iz repera $Oe$ prelazimo u reper $Qf$. Dodatno, neka je
$C$ matrica prelaska iz repera $Oe$ u reper $Qf$. Tada:
    $$[X]_{Qf}  =[\overrightarrow{QX}]_f =[\overrightarrow{QO}]_f+[\overrightarrow{OX}]_e =[O]_{Qf}+C\cdot[\overrightarrow{OX}]_e =[O]_{Qf}+C\cdot[\overrightarrow{X}]_{Oe}$$
    Ako je $[X]_{Oe}=x'$, $[X]_{Qf}=x$ i $[O]_{Qf}=q$, tada dobijamo $x=q+C\cdot x'$.
    \par
    \subsection{Transformacije koordinata u ON bazama ravni}
    \label{subsec:pitanje_9}
    Slučaj baza iste orijentacije, matrica rotacije, osobine matrice
    rotacije, slučaj baza različite orijentacije, matrica refleksije,
    osobine matrica refleksije, grupe $SO(2)$ i $O(2)$.\\
    \textbf{Odgovor:}\\
    \textit{I slučaj:} Neka su baze $Oe$ i $Qf$ ON i iste orijentacije. Neka je $e=(\overrightarrow{e_1},\overrightarrow{e_2})$
    i $f=(\overrightarrow{f_1},\overrightarrow{f_2})$ i neka je $\angle(\overrightarrow{e_1}\overrightarrow{f_1})=\phi$, $\phi\in[0,2\cdot\pi)$.
    Tada je:\\
$\overrightarrow{f_1}=\overrightarrow{e_1}\cdot\cos(\phi)+\overrightarrow{e_2}\cdot\sin(\phi)$\\
$\overrightarrow{f_2}=\overrightarrow{e_1}\cdot\cos(\phi+\frac{\pi}{2})+\overrightarrow{e_2}\cdot\sin(\phi+\frac{\pi}{2})=\overrightarrow{e_1}\cdot(-\sin(\phi))+\overrightarrow{e_2}\cdot\cos(\phi)$.\\
    Odavde dobijamo formulu za istu orijentaciju: $x=q+\begin{bmatrix}
    \cos(\phi) & -\sin(\phi) \\
    \sin(\phi) & \cos(\phi)
\end{bmatrix}\cdot x'$.\\
    Matrica koju smo dobili je matrica rotacije oko $O$ za ugao $\phi$.\\
    Osobine matrice rotacije:\\
$(R_\phi)^{-1}=R_{-\phi}=(R_\phi)^T$\\
$det(R_\phi)=1$\\
$R_\phi\cdot R_\psi=R_{\phi+\psi}=R_\psi\cdot R_\phi$\\
    \textit{II slučaj:} Neka su baze $Oe$ i $Qf$ ON i različite orijentacije. Neka je $e=(\overrightarrow{e_1},\overrightarrow{e_2})$
    i $f=(\overrightarrow{f_1},\overrightarrow{f_2})$ i neka je $\angle(\overrightarrow{e_1}\overrightarrow{f_1})=\phi$, $\phi\in[0,2\cdot\pi)$.
    Tada je:\\
$\overrightarrow{f_1}=\overrightarrow{e_1}\cdot\cos(\phi)+\overrightarrow{e_2}\cdot\sin(\phi)$\\
$\overrightarrow{f_2}=\overrightarrow{e_1}\cdot\cos(\phi-\frac{\pi}{2})+\overrightarrow{e_2}\cdot\sin(\phi-\frac{\pi}{2})=\overrightarrow{e_1}\cdot\sin(\phi)+\overrightarrow{e_2}\cdot(-\cos(\phi))$.\\
    Odavde dobijamo formulu za istu orijentaciju: $x=q+\begin{bmatrix}
    \cos(\phi) & \sin(\phi)  \\
    \sin(\phi) & -\cos(\phi)
\end{bmatrix}\cdot x'$.\\
    Matrica koju smo dobili je matrica refleksije u odnosu na pravu $p_0$ koja
    prolazi kroz $O$ i gradi sa $\overrightarrow{e_1}$ ugao $\frac{\phi}{2}$.\\
    Osobine matrice refleksije:\\
$(S_\phi)^{-1}=(S_\phi)^T$\\
$(S_\phi)^{2}=E$\\
$det(S_\phi)=-1$\\
$S_\phi\cdot S_\psi=R_{\phi-\psi}$\\
    Specijalna ortogonalna grupa reda 2: $SO(2)=\{R_\phi\mid\phi\in\mathbb{R}\}$.\\
    Ortogonalna grupa reda 2: $O(2)=SO(2)\cup\{S_\phi\mid\phi\in\mathbb{R}\}=\{A\in Gl_2(\mathbb{R})\mid A^{-1}=A^T\}$.
    \par
    \subsection{Afina preslikavanja}
    Definicija, pasivna i aktivna tačka gledišta, osobine afinih
    preslikavanja (T5.2) (dokaz), odredjenost afinog
    preslikavanja sa tri para odgovarajućih tačaka,
    predstavljanje afinih preslikavanja ravni 3 x 3 matricama,
    translacija, rotacija (oko O i oko proizvoljne tačke), refleksija
    u odnosu na pravu, skaliranje, smicanje, primeri.
    \\
    \textbf{Odgovor:}\\
    Definicija afinih preslikavanja: Neka je $\overline{f}: \mathbb{V}\rightarrow\mathbb{V}$
    linearno preslikavanje vektorskog prostora koji je pridružen prostoru
    tačaka $\mathbb{E}$. Afino preslikavanje $f: \mathbb{E}\rightarrow\mathbb{E}$
    je preslikavanje tačaka koje je indukovano sa $\overline{f}$ u smislu
    da važi $f(M)=M'\land f(N)=N' \iff \overline{f}(\overrightarrow{MN})=\overrightarrow{M'N'}$.\\
    Pasivna tačka gledišta je kada fiksiramo tačku i posmatramo je iz više repera.\\
    Aktivna tačka gledišta je kada fiksiramo reper i posmatramo neku tačku pre i posle afinog preslikavanja.\\
    Teorema: Svako afino preslikavanje $f$, u fiksiranom reperu $Oe$, ima oblik:\\
$[f(M)]_{Oe}=[\overline{f}]_{e}\cdot[M]_{Oe}+[f(O)]_{Oe}$.\\
    Dokaz:
    \begin{align*}
        [f(M)]_{Oe}-[f(O)]_{Oe} & =[\overrightarrow{Of(M)}]_e-[\overrightarrow{Of(O)}]_e =[\overrightarrow{Of(M)}]_e+[\overrightarrow{f(O)O}]_e      \\
                                & =[\overrightarrow{f(O)f(M)}]_e=[\overline{f}(\overrightarrow{OM})]_e =[\overline{f}]_e\cdot[\overrightarrow{OM}]_e \\
                                & =[\overline{f}]_e\cdot[M]_{Oe}
    \end{align*}
    Napomena: Kolone matrice $[\overline{f}]_e$ su zapravo slike baznih vektora baze $e$ ($\overline{f}(\overrightarrow{e_i})$).\\
    Osobine afinih preslikavanja ravni (\textit{T5.2}):\\
    Preslikaju prave u prave:\\
    S obzirom da je svako afino preslikavanje indukovano linearnim preslikavanjem
    vektorskog prostora, a linearno preslikavanje vektorskog prostora
    čuva kolinearnost, tada i afino preslikavanje čuva kolinearnost, odnosno slika prave u prave.\\
    Čuvaju razmeru kolinearnih duži:\\
    Analogno prošloj osobini.\\
    Čuvaju paralelnost pravih:\\
    Ako se posmatraju po dve tačke na dve paralelne prave,
    dobijamo 2 kolinearna vektora. Odavde se analogno završava dokaz.\\
    Odnos površina slike i originala jednak je $|det(a_{ij})|$:\\
    Napomenimo da je $a_{ij}$ zapravo matrica $[\overline{f}]_e$.
    Zbog jednostavnijeg zapisa i jer je reper fiksiran ćemo izostaviti reper iz zapisa.\\
    Neka je $f(O)=O'$, $f(A)=A'$ i $f(B)=B'$. Površina trougla
$OAB$ je $\frac{1}{2}\cdot |det(\overrightarrow{OA}\overrightarrow{OB})|$,
    a površina trougla $O'A'B'$ je $\frac{1}{2}\cdot |det(\overrightarrow{O'A'}\overrightarrow{O'B'})|$.
    Takođe su $\overrightarrow{O'A'}=a_{ij}\cdot \overrightarrow{OA}$ i $\overrightarrow{O'B'}=a_{ij}\cdot \overrightarrow{OB}$.
    Za kraj, koristimo svojstvo $det(A\cdot B)=det(A)\cdot det(B)$.
    \begin{align*}
        1           & =\frac{det(\overrightarrow{OA}\ \overrightarrow{OB})}{det(\overrightarrow{OA}\ \overrightarrow{OB})}                       \\
        det(a_{ij}) & =\frac{det(a_{ij}\cdot\overrightarrow{OA}\ a_{ij}\cdot\overrightarrow{OB})}{det(\overrightarrow{OA}\ \overrightarrow{OB})} \\
        det(a_{ij}) & =\frac{det(\overrightarrow{O'A'}\ \overrightarrow{O'B'})}{det(\overrightarrow{OA}\ \overrightarrow{OB})}                   \\
    \end{align*}
    Čuvaju centar mase i baricentričke koordinate:\\
    Ovo sledi direktno iz čuvanja razmere.\\
    Preslikavanja za koje važi $det(a_{ij})>0$ čuvaju orijentaciju,
    inače menjaju:\\
    Po algebarskoj definiciji, dve baze su iste orijentacije ako
    je determinanta matrice prelaska iz jedne u drugu veća od
    nule. Ovo je aktivna varijanta ove definicije.\\
    Ako su date tri nekolinearne tačke i njihove slike
    može se odrediti jedinstveno afino preslikavanje ravni.\\
    Matrica translacije:
$\mathcal{T}_{\overrightarrow{t}}=\begin{bmatrix}
    1 & 0 & x_t \\
    0 & 1 & y_t \\
    0 & 0 & 1
\end{bmatrix}$, gde je $\overrightarrow{t}=(x_t,y_t)$.\\
    Matrica rotacije oko $O$:
$\mathcal{R}_\phi=\begin{bmatrix}
    \cos(\phi) & -\sin(\phi) & 0 \\
    \sin(\phi) & \cos(\phi)  & 0 \\
    0          & 0           & 1
\end{bmatrix}$, gde je $\phi$ ugao za koji se rotira.\\
    Rotacija oko proizvoljne tačke $M$:
$\mathcal{T}_{\overrightarrow{OM}}\circ\mathcal{R}_\phi\circ\mathcal{T}_{\overrightarrow{MO}}$.\\
    Matrica refleksija u odnosu na pravu koja prolazi kroz $O$:
$\mathcal{S}_{p_0}=\begin{bmatrix}
    cos(\phi) & sin(\phi)  & 0 \\
    sin(\phi) & -cos(\phi) & 0 \\
    0         & 0          & 1
\end{bmatrix}$, gde je $\frac{\phi}{2}$ ugao između $x$ ose i $p_0$.\\
    Refleksija u odnosu na pravu koja prolazi kroz tačku $M$ se dobija analogno kao rotacija.\\
    Matrica skaliranja sa centrom u tački $O$:
$\mathcal{H}_{\lambda_1,\lambda_2}=\begin{bmatrix}
    \lambda_1 & 0         & 0 \\
    0         & \lambda_2 & 0 \\
    0         & 0         & 1
\end{bmatrix}$, gde je $\lambda_1$ koeficijent skaliranja u pravcu
    prvog baznog vektora, a $\lambda_2$ koeficijent skaliranja u pravcu
    drugog baznog vektora.\\
    Skaliranje sa centrom u proizvoljnoj tački analogno rotaciji.\\
    Smicanje preslikava kvadrat u pravougaonik iste površine.\\
    Matrice smicanja:\\
$\mathcal{S}_x(\lambda)=\begin{bmatrix}
    x & \lambda & 0 \\
    0 & y       & 0 \\
    0 & 0       & 1
\end{bmatrix}$, gde je $\lambda$ koeficijent smicanja u pravcu $x$ ose.\\
$\mathcal{S}_y(\lambda)=\begin{bmatrix}
    x & \lambda & 0 \\
    0 & y       & 0 \\
    0 & 0       & 1
\end{bmatrix}$, gde je $\lambda$ koeficijent smicanja u pravcu $y$ ose.

    \par
    \subsection{Izometrije ravni i prostora}
    Defincija, direktne i indirekne izometrije, ortogonalne
    matrice i njihova determinanta, opis afinih transformacija
    ravni (pitanje 9), rotacija oko prave u prostoru, rotacije
    oko koordinatnih osa, Rodrigezova formula, refleksija u
    odnosu na ravan (računanje), Prva Ojlerova teorema,
    Ojlerovi uglovi, Druga Ojlerova teorema, veza sopstvenih i
    svetskih rotacija - smisao, slučaj zaključanog žiroskopa.\\
    \textbf{Odgovor:}\\
    Definicija izometrije: Preslikavanje koje čuva dužine u prostoru
$\mathbb{E}$ proizvoljne dimenzije.\\
    Direktna izometrija čuva orijentaciju, to su kretanja.\\
    Indirektna izometrija menja orijentaciju.\\
    Ortogonalna matrica je matrica $A$ za koju važi $A\cdot A^T=E$.\\
    Za determinantu ortogonalne matrice važi:\\
$1=det(E)=det(A\cdot A^T)=det(A)\cdot det(A^T)=det(A)\cdot det(A)=det(A)^2$.\\
    \hyperref[subsec:pitanje_9]{pitanje 9}\\
    Matrica rotacje oko $x$ ose: $\mathcal{R}_x(\phi)=\begin{bmatrix}
    1 & 0          & 0           \\
    0 & \cos(\phi) & -\sin{\phi} \\
    0 & \sin(\phi) & \cos(\phi)
\end{bmatrix}$\\[0.5cm]
    Matrica rotacje oko $y$ ose: $\mathcal{R}_y(\phi)=\begin{bmatrix}
    \cos(\phi) & 0 & -\sin{\phi} \\
    0          & 1 & 0           \\
    \sin(\phi) & 0 & \cos(\phi)
\end{bmatrix}$\\[0.5cm]
    Matrica rotacje oko $z$ ose: $\mathcal{R}_z(\phi)=\begin{bmatrix}
    \cos(\phi) & -\sin{\phi} & 0 \\
    \sin(\phi) & \cos(\phi)  & 0 \\
    0          & 0           & 1
\end{bmatrix}$\\[0.5cm]
    Rotacija oko proizvoljne prave koja sadrži koordinatni početak (Rodrigezova formula):\\
    Neka su date tačka $M$ i prava $p$ koja sadrži koordinatni početak $O$ i ugao $\phi$ za koji rotiramo tačku $M$ oko prave $p$.
    Neka je jedinični vektor pravca prave $p$ vektor $\overrightarrow{p}$.
    Posmatrajmo ravan $\alpha$ koja sadrži tačku $M$ i njen normalni vektor je vektor $\overrightarrow{p}$.
    Neka je $\alpha\cap p =\{O'\}$ i neka su $\overrightarrow{x}=\overrightarrow{OM}$, $\overrightarrow{x_1}=\overrightarrow{OO'}$ i $\overrightarrow{x_2}=\overrightarrow{O'M}$.
    \begin{align*}
        \overrightarrow{x}                        & =\overrightarrow{x_1}+\overrightarrow{x_2} = \lambda\cdot\overrightarrow{p}+\overrightarrow{x_2}   \\
        \overrightarrow{p}\cdot\overrightarrow{x} & =\lambda\cdot\overrightarrow{p}\cdot\overrightarrow{p}+\overrightarrow{x_2}\cdot\overrightarrow{p} \\
                                                  & =\lambda\cdot|\overrightarrow{p}|^2                                                                \\
                                                  & =\lambda                                                                                           \\
    \end{align*}
    Dobijamo:\\
$\overrightarrow{x_1}=\lambda\cdot \overrightarrow{p}=\overrightarrow{p}\cdot\lambda=\overrightarrow{p}\cdot(\overrightarrow{p}^T\cdot \overrightarrow{x})=(\overrightarrow{p}\cdot \overrightarrow{p}^T)\cdot \overrightarrow{x}$.
    Treća jednakost važi jer se skalarni proizvod može preko koordinata zapisati kao $\overrightarrow{u}^T\cdot \overrightarrow{v}$.\\
$\overrightarrow{x_2}=\overrightarrow{x}-\overrightarrow{x_1}=E\cdot\overrightarrow{x}-(\overrightarrow{p}\cdot \overrightarrow{p}^T)\cdot \overrightarrow{x}=(E-\overrightarrow{p}\cdot \overrightarrow{p}^T)\cdot \overrightarrow{x}$.\\
    Neka je $M'$ tačka koja se dobija rotacijom tačke $M$ oko prave $p$ za ugao $\phi$.
    Možemo da posmatramo ortogonalnu bazu $(\overrightarrow{p}$,$\overrightarrow{x_2}$,$\overrightarrow{p}\times\overrightarrow{x_2})$.\\
    Tada $\overrightarrow{O'M'}=\cos(\phi)\cdot\overrightarrow{x_2}+\sin(\phi)(\overrightarrow{p}\times\overrightarrow{x_2})=\cos(\phi)\cdot\overrightarrow{x_2}+\sin(\phi)(\overrightarrow{p}\times(\overrightarrow{x_1}+\overrightarrow{x_2}))=\cos(\phi)\cdot\overrightarrow{x_2}+\sin(\phi)(\overrightarrow{p}\times\overrightarrow{x})$.
    Na kraju dobijamo:
    $$\overrightarrow{OM'}=\overrightarrow{OO'}+\overrightarrow{O'M'}=(\overrightarrow{p}\cdot \overrightarrow{p}^T)\cdot \overrightarrow{x}+\cos(\phi)\cdot(E-\overrightarrow{p}\cdot \overrightarrow{p}^T)\cdot \overrightarrow{x}+\sin(\phi)(\overrightarrow{p}_\times\cdot\overrightarrow{x})$$
    Odavde dobijamo matricu: $[R_p(\phi)]_e=p\cdot p^T+\cos(\phi)\cdot(E-p\cdot p^T)+\sin(\phi)\cdot\overrightarrow{p}_\times$ koju nazivamo Rodrigezova formula.\\
    Refleksija u odnosu na ravan koja sadrži koordinatni početak:\\
    Neka je $\alpha$ ravan koja prolazi kroz $O$ zadata jediničnim
    normalnim vektorom $\overrightarrow{p}$ i neka je data tačka $M$
    na koju hoćemo da primenimo refleksiju u odnosu na ravan i neka se
    tom refleksijom dobija tačka $M'$. Neka je $\alpha\cap MM'=\{O'\}$.\\
$\overrightarrow{OM'}=\overrightarrow{OM}+\overrightarrow{MM'}=\overrightarrow{OM}-2\cdot \overrightarrow{O'M}$.
    Analogno sa pitanjem za Rodrigezovu formulu:\\
$\overrightarrow{O'M}=\lambda\cdot\overrightarrow{p}$\\
$\overrightarrow{p}\cdot\overrightarrow{O'M}=\overrightarrow{p}\cdot(\overrightarrow{O'O}+\overrightarrow{OM})=\overrightarrow{p}\cdot\overrightarrow{OM}=\lambda$\\
$\overrightarrow{O'M}=\overrightarrow{p}\cdot (\overrightarrow{p}^T\cdot\overrightarrow{OM})$\\
$\overrightarrow{OM'}=\overrightarrow{OM}-2\cdot(\overrightarrow{p}\cdot \overrightarrow{p}^T)\cdot\overrightarrow{OM}$\\
    Odavde dobijamo matricu: $[S_\alpha]_e=E-2\cdot p\cdot p^T$.\\
    Prva Ojlerova teorema: Svako kretanje koje ima fiksiranu tačku
$O'$ se može predstaviti kao rotacija oko neke orijentisane
    prave koja sadrži tačku $O'$.\\
    Druga Ojlerova teorema: Svako kretanje koje fiksira koordinatni
    početak se može izraziti kao kompozicija tri sopstvene rotacije
    oko koordinatnih osa:\\
$R_{Ox_2}(\phi)\circ R_{Oy_1}(\theta)\circ R_{Oz}(\psi)$ gde su $\phi,\psi\in[0,2\cdot\pi),\theta\in[-\frac{\pi}{2},\frac{\pi}{2}]$.\\
    Uglovi $\phi$,$\theta$,$\psi$ se nazivaju Ojerovi uglovi. Prvom rotacijom se $Oxyz$ slika u $Ox_1y_1z_1$,
    drugom rotacijom u $Ox_2y_2z_2$ i trećom rotacijom u $Ox_3y_3z_3$.\\
    Veza sopstvenih i svetskih koordinata je:\\
$[R_{Ox_2}(\phi)\circ R_{Oy_1}(\theta)\circ R_{Oz}(\psi)]_e = R_{Oz}(\psi)\circ R_{Oy}(\theta)\circ R_{Ox}(\phi)$.\\
    Zaključavanje žiroskopa se javlja kada se ravni $Oxy$ i $Oy_3z_3$ poklope. Tada
    je $|Oz|=|Ox_3|$ zbog čega imamo jedan stepen slobode manje.
    \par
    \subsection{Kvaternioni}
    Definicija, sabiranje, množenje, osobine, inverzni
    kvaternion, primeri, konjugacija kvaternionom i izometrije.
    \\
    \textbf{Odgovor:}\\
    Definicija: $\mathbb{H}=\{x\cdot i+y\cdot j+z\cdot k+w\mid x,y,z,w\in\mathbb{R}\}$\\
    Sabiranje: normalno.\\
    Množenje: $i\cdot j=k=-j\cdot i$ i $i^2=j^2=k^2=-1$.\\
    Ako posmatramo kvaternion $q=x\cdot i +y\cdot j +z\cdot k+w$:\\
    Realni deo je $\mathcal{R}(q)=w$.\\
    Imaginarni deo je $\mathcal{I}(q)=x\cdot i +y\cdot j +z\cdot k=\overrightarrow{v}$.\\
    Važi $q=x\cdot i +y\cdot j +z\cdot k+w=\mathcal{I}(q)+\mathcal{R}(q)=[(x,y,z),w]=[\overrightarrow{v},w]$.\\
    Osobine:\\
$q+q_1=[\overrightarrow{v},w]+[\overrightarrow{v_1},w_1]=[\overrightarrow{v}+\overrightarrow{v_1},w+w_1]$\\
$q\cdot q_1=[\overrightarrow{v},0]\cdot[\overrightarrow{v_1},0]=[\overrightarrow{v}\times\overrightarrow{v_1},-\langle\overrightarrow{v},\overrightarrow{v_1}\rangle]$\\
$q\cdot q_1=[\overrightarrow{v},w_1]\cdot[\overrightarrow{v_1},w_1]=[\overrightarrow{v}\times\overrightarrow{v_1}+w\cdot \overrightarrow{v_1}+w_1\cdot\overrightarrow{v},w\cdot w_1-\langle\overrightarrow{v},\overrightarrow{v_1}\rangle]$\\
    Napomena: $\langle\overrightarrow{v},\overrightarrow{u}\rangle$ je drugi zapis za skalarni proizvod.\\
    Konjugovani kvaternion: $\overline{q}=-x\cdot i -y\cdot j -z\cdot k+w$.\\
    Inverzni kvaternion: $q\cdot q^{-1}=1$.\\
    Osobina: $q\cdot \overline{q}=|q|^2 \implies q^{-1}=\frac{\overline{q}}{|q|^2}$.\\
    Konjugacije kvaternionom:\\
    Svaki kvaternion različit od nule određuje konjugaciju: $C_q=q\cdot p \cdot q^{-1}$.
    Dve konjugacije su iste ako i samo ako važi $q=\lambda\cdot h$, gde je $\lambda\in\mathbb{R}$.\\
    Konjugacija $C_q$ je izometrija prostora $\mathcal{I}\mathbb{H}\cong\mathbb{R}^3$.\\
    Konjugacija kvaternionom $q=[\overrightarrow{v}\cdot\sin(\frac{\alpha}{2}),\cos(\frac{\alpha}{2})]$
    je rotacija oko jediničnog vektora $\overrightarrow{v}$
    za ugao $\alpha$ u pozitivnom smeru.
    \par
    \subsection{Afina geometrija ravni}
    Predstavljanje prave, razne jednačine pravih, normalni
    vektor prave, vektor pravca prave, presek dve prave zadate
    jednačinama, presek pravih zadatih tačkom i vektorom pravca
    (dokaz), presek dve duži, parametarsko zadavanje trougla,
    paralelograma.
    \\
    \textbf{Odgovor:}\\
    Implicitna jednačina prave: $a\cdot x+b\cdot y + c = 0$.\\
    Parametarska jednačina prave: $M(t)=P+t\cdot \overrightarrow{p}$.\\
    Iz implicitnog zapisa možemo da pročitamo normalni vektor $\overrightarrow{n_p}=(a,b)$.\\
    Iz parametarskog zapisa možemo da pročitamo vektor pravca $\overrightarrow{p}$.\\
    Presek pravih se dobija rešavanjem dve jednačine prave sa dve nepoznate.\\
    Duž $AB$ možemo da zadamo sa parametarski sa: $M(t)=A+t\cdot \overrightarrow{AB}, t\in[0,1]$.\\
    Presek duži: možemo da odredimo gde se nalazi presek rešavanjem jednačina,
    a zatim da proverimo da li je ta tačka između tačaka $A_1$, $B_1$ i $A_2$, $B_2$.\\
    Trougao možemo da zadamo sa: $M(t_1,t_2)=A+t_1\cdot \overrightarrow{AB}+t_2\cdot \overrightarrow{AC}, t_1, t_2\in[0,1], t_1+t_2\leq1$.
    Paralelogram možemo da zadamo sa: $M(t_1,t_2)=A+t_1\cdot \overrightarrow{AB}+t_2\cdot \overrightarrow{AC}, t_1, t_2\in[0,1]$.
    \par
    \subsection{Ravan u prostoru}
    Zadavanje ravni tačkom i normalnim vektorom (dokaz),
    skiciranje ravni, specijalni slučajevi ravni, parametarska
    jednačina ravni, jednačina ravni kroz 3 tačke, primeri.
    \\
    \textbf{Odgovor:}\\
    Zadavanje ravni tačkom i normalnim vektorom:\\
    Neka je dat normalni vektor $\overrightarrow{n_\alpha}=(a,b,c)$
    i tačka $P=(x_1,y_1,z_1)$ koja pripada ravni.
    Posmatrajmo tačku sa ravni $A=(x,y,z)$. Važi da
    je $\overrightarrow{PA}\cdot\overrightarrow{n_\alpha}=0$.
    Odavde:
$\overrightarrow{PA}\circ\overrightarrow{n_\alpha}=(x-x_1,y-y_1,z-z_1)\circ(a,b,c)=a\cdot x+b\cdot y+c\cdot z-(a\cdot x_1+b\cdot y_1+c\cdot z_1)=0$.
    Ako obeležimo $d=-(a\cdot x_1+b\cdot y_1+c\cdot z_1)$ dobijamo $a\cdot x+b\cdot y+c\cdot z+d=0$.
    Skiciranje ravni može da se vrši tako što se prvo izračunaju i označe preseci sa osama.\\
    Parametarska jednačina ravni: $M(t,s)=A+t\cdot\overrightarrow{v}+s\cdot\overrightarrow{u}$,
    gde je $A$ tačka koja pripada ravni, a $\overrightarrow{v}$ i $\overrightarrow{u}$
    su vektori paralelni sa ravni.\\
    Jednačina ravni kroz tri tačke:\\
    Ako su date nekolinearne tačke $A$, $B$ i $C$, tada
    je jednačine ravni: $M(t,s)=A+t\cdot\overrightarrow{AB}+s\cdot\overrightarrow{AC}$.
    Ako su tačke kolinearna ne može se jednoznačno odrediti ravan koja ih sadrži.
    \par
    \subsection{Prava u prostoru}
    Parametarska i ”kanonska” jednačina prave, prava AB i
    podela duži AB na jednake delove, prava kao presek dve
    ravni, pramen ravni, primeri.
    \\
    \textbf{Odgovor:}\\
    Parametarska jednačina: $M(t)=A+t\cdot\overrightarrow{p}$,
    gde je $A$ tačka na pravoj, a $\overrightarrow{p}$ je vektor
    pravca.\\
    "Kanonska" jednačina je pod navodnicima jer nešto što je kanonsko bi
    trebalo da bude jedinstveno, ali ova jednačina nije jedinstvena.\\
    Kanonska jednačina prave: $\frac{x-x_0}{p_x}=\frac{y-y_0}{p_y}=\frac{z-z_0}{p_z}=t$.\\
    Ako želimo da podelimo duž $AB$ na $n$ delova i
    želimo da nađemo $m$-tu tačku podele to radimo sa:
$T_m=A+\overrightarrow{AB}\cdot\frac{m}{n}$, važi da je
$T_0=A$ i $T_n=B$.\\
    Prava kao presek dve ravni se dobija kao rešenje sistema
    dve jednačine od tri nepoznate. Može da nema rešenja, ravni su paralelne,
    ima beskonačno mnogo rešenja koje će formirati prava, ravni se seku po pravoj ili
    ima beskonačno mnogo rešenja koje će formirati ravan, ravni se poklapaju.\\
    Ako se dve ravni seku u pravoj, njihovom linearnom kombinacijom
    se dobija svaka ravan koja sadrži tu pravu i ovo se naziva
    pramen ravni. Opšti primer:\\
$\alpha: a_1\cdot x+b_1\cdot y+c_1\cdot z+d_1=0$\\
$\beta: a_2\cdot x+b_2\cdot y+c_2\cdot z+d_2=0$\\
    Tada je pramen ravni:\\
$\lambda_1\cdot(a_1\cdot x+b_1\cdot y+c_1\cdot z+d_1)+\lambda_2\cdot (a_2\cdot x+b_2\cdot y+c_2\cdot z+d_2)=0$.
    \par
    \subsection{Međusobni položaji pravih i ravni}
    \label{subsec:pitanje_16}
    Međusobni položaji dve prave (zadatih tačkom i vektorom
    pravca) u prostoru, međusobni položaj prave i ravni,
    međusobni položaji dve ravni, mimoilazne prave, zajednička
    normala - T4.3 (dokaz), rastojanje mimoilaznih pravih -
    T4.4 (dokaz).
    \\
    \textbf{Odgovor:}\\
    Dve prave mogu da budu paralelne, mimoilazne,
    da se poklapaju i da se seku. Možemo da rešavamo
    dve jednačine sa dve nepoznate. Ako postoji
    jedno rešenje onda se seku. Ako postoji beskonačno
    mnogo rešenja onda se poklapaju. Ako nema rešenja
    a vektori su kolinearni onda su prave paralelne.
    Ako ništa od prethodnog ne važi onda su mimoilazne.\\
    Prava i ravan mogu da budu paralelne, da se seku
    ili da ravan sadrži pravu. Ako sistem jednačina
    ima jedinstveno rešenje, onda se seku, ako ima
    beskonačno mnogo rešenja onda ravan sadrži pravu,
    inače su paralelne.
    Dve ravni mogu da se poklapaju, budu paralelne ili da se seku.
    Analogni su slučajevi pravoj i ravni.\\
    Mimoilazne prave su prave preko kojih ne može da se
    postavi ravan.\\
    T4.3 Mimoilazne prave $p$ i $q$ imaju jedinstvenu
    zajedničku normalu:\\
    Dokaz:\\
    Neka su date prave $p$ i $q$. Vektor pravca
    prave $n$ koja je normalna na obe prave će
    biti $\overrightarrow{p}\times\overrightarrow{q}$ gde su
$\overrightarrow{p}$ i $\overrightarrow{q}$ vektori
    pravca pravih $p$ i $q$. Sve prave čiji su
    vektori pravca ortogonalni
    i sa $\overrightarrow{p}$ i sa $\overrightarrow{q}$
    i koje seku pravu $p$ će
    pripadati ravni $\alpha$ koja se konstruiše
    jednom tačkom sa prave $p$ i vektorima $\overrightarrow{p}$
    i $\overrightarrow{p}\times\overrightarrow{q}$.
    Analogno možemo da konstruišemo ravan $\beta$
    koja će sadržati sve prave čiji su
    vektori pravca ortogonalni
    i sa $\overrightarrow{p}$ i sa $\overrightarrow{q}$
    i koje seku pravu $q$. Presek ove dve ravni
    će biti prava $n$ koja je jedinstvena jer
    su $p$ i $q$ mimoilazne. Prava $n$ je normalna
    na prave $p$ i $q$ jer je njen koeficijent pravca
$\overrightarrow{p}\times \overrightarrow{q}$ i seče
    ih po konstrukciji.\\
    T4.4 Rastojanje između mimoilaznih pravih je:
$\frac{|[\overrightarrow{p},\overrightarrow{q},\overrightarrow{PQ}]|}{|p\times q|}$ gde su
$P$ i $\overrightarrow{p}$ tačka i vektor pravca prave $p$,
    a $Q$ i $\overrightarrow{q}$ tačka i vektor pravca prave $q$.\\
    Dokaz:\\
    U brojiocu je zapravo zapremina paralelopipeda, a u
    imeniocu je površina baze pa se deljenjem tačno dobija
    visina koju tražimo.
    \subsection{Projekcije}
    Normalna projekcija na ravan $Oxy$ (je linearno preslikavanje),
    normalna projekcija na \\proizvoljnu ravan (dokaz),
    centralna projekcija, formule centralne projekcije iz
$C(0,0,0)$ na ravan $z=f$, kartografske projekcije,
    definicija i formule stereografska projekcija.
    \\
    \textbf{Odgovor:}\\
    Matrica normalne projekcije na $Oxy$ ravan:
$\begin{bmatrix}
    x' \\
    y' \\
    z' \\
    1
\end{bmatrix}=\begin{bmatrix}
    1 & 0 & 0 & 0 \\
    0 & 1 & 0 & 0 \\
    0 & 0 & 1 & 0 \\
    0 & 0 & 0 & 1
\end{bmatrix}=\begin{bmatrix}
    x \\
    y \\
    z \\
    1
\end{bmatrix}$.\\
    Matrica normalne projekcije na proizvoljnu ravan koja sadrži koordinatni početak:
    Neka je $\alpha$ ravan koja prolazi kroz $O$ zadata jediničnim
    normalnim vektorom $\overrightarrow{p}$ i neka je data tačka $M$
    koju hoćemo da projektujemo na ravan i neka se
    tom projekcijom dobija tačka $M'$.\\
$\overrightarrow{OM'}=\overrightarrow{OM}-\overrightarrow{M'M}$.
$\overrightarrow{M'M}=\lambda\cdot\overrightarrow{p}$\\
$\overrightarrow{p}\cdot\overrightarrow{M'M}=\overrightarrow{p}\cdot(\overrightarrow{M'O}+\overrightarrow{OM})=\overrightarrow{p}\cdot\overrightarrow{OM}=\lambda$\\
$\overrightarrow{M'M}=\overrightarrow{p}\cdot (\overrightarrow{p}^T\cdot\overrightarrow{OM})$\\
$\overrightarrow{OM'}=\overrightarrow{OM}-(\overrightarrow{p}\cdot \overrightarrow{p}^T)\cdot\overrightarrow{OM}$\\
    Odavde dobijamo matricu: $x'=(E-p\cdot p^T)\cdot x$.\\
    Centralna projekcija: Za centralnu projekciju je potrebno
    da fiksiramo ravan na koju projektujemo i tačku u odnosu
    na koju projektujemo (iz koje posmatramo).\\
    Centralna projekcija u odnosu na tačku $C(0,0,0)$ na ravan $z=d$:
    Posmatrajmo proizvoljnu tačku $M(x,y,z)$. Neka je njena projekcija tačka $M'(x',y',z')$.
    Tada je $x'=\frac{d}{z}\cdot x$, $y'=\frac{d}{z}\cdot y$ i $z'=\frac{d}{z}\cdot z=d$.
    Ovo možemo izraziti matrično preko homogenih koordinata:\\
    Neka je $x'=\frac{x_1'}{x_4'}$, $y'=\frac{x_2'}{x_4'}$ i $z'=\frac{x_3'}{x_4'}$.
    Tada se centralna projekcija može predstaviti matricom:
$\begin{bmatrix}
    x_1' \\
    x_2' \\
    x_3' \\
    x_4'
\end{bmatrix}=\begin{bmatrix}
    d & 0 & 0 & 0 \\
    0 & d & 0 & 0 \\
    0 & 0 & d & 0 \\
    0 & 0 & 1 & 0
\end{bmatrix}=\begin{bmatrix}
    x \\
    y \\
    z \\
    1
\end{bmatrix}$.\\
    Kartografske projekcije projektuju zakrivljene površine na
    ravan.\\
    Stereografska projekcija je specijalan slučaj centralne
    projekcije kojom se jedinična sfera projektuje na ravan.
    centar projekcije je neka tačka sfere, a ravan na koju se
    projektuje tangira sferu.\\
    Izvođenje formula:\\
    Neka je $C=(0,0,1)$ i $z=-1$. Posmatrajmo tačku na lopti
$M=(x,y,z)$ i pravu $p(\lambda)=C+\lambda\cdot\overrightarrow{CM}$.
    Važi $x'=\lambda\cdot x$, $y'=\lambda\cdot y$ i $z'=\lambda\cdot z+\lambda=-1$.
    Odavde dobijamo $\lambda=\frac{1}{1-z}$. Na kraju dobijamo formule stereografske
    projekcije:
$x'=\frac{x}{1-z}$, $y'=\frac{y}{1-z}$ i $z'=-1$.
    \par
    \subsection{Uglovi između pravih i ravni}
    Merenje uglova (stepeni, radijani, nagib u procentima),
    ugao između 2 prave, ugao između prave i ravni,
    ugao između dve ravni, ugao između pljosni tetraedra
    (na dva načina), računski primeri.\\
    \textbf{Odgovor:}\\
    Veza stepena i radijana: $180^\circ=\pi$.\\
    Radijan predstavlja odnos dužine luka i dužine poluprečnika.\\
    Merenje nagiba u procentima: Ako merimo ugao između hipotenuze
    i jedne katete i dužina te katete je $a$, a dužina druge
    katete je $b$, tada će nagib biti $\frac{b}{a}\cdot 100\%$. Dakle,
    ako je dužina jedne hipotenuze $1$ metar, a dužina druge
    hipotenuze $5$ centimetara, tada će nagib hipotenuze biti $5\%$.
    Ugao između dve prave se određuje kao ugao između vektora pravca
    tih pravi.\\
    Ugao između prave i ravni se određuje kao $\frac{\pi}{2}-\angle(\overrightarrow{n_\alpha}\overrightarrow{p})$
    gde je $\overrightarrow{n_\alpha}$ normalni vektor ravni, a $\overrightarrow{p}$ vektor pravca prave.\\
    Ugao između dve ravni je ugao između njihovih normalnih vektora.
    Ugao između pljosni tetraedra:\\
    Prvi način:\\
    Posmatrajmo jediničnu kocku $ABCDA'B'C'D'$.
    Tačke $ACB'D'$ formiraju pravilni tetraedar. Lako možemo
    da odredimo koordinate ovih tačaka.\\
    Drugi način:\\
    Posmatrajmo pravilni jedinični tetraedar $ABCD$. Označimo sredinu
    stranice $AB$ sa $M$. Izdvojimo trougao $MDC$. Zanima nas
    ugao $CMD$. Neka je $|MD|=\frac{\sqrt{3}}{2}$, $|DC|=1$ i $|MC|=\frac{\sqrt{3}}{2}$.
    Preko kosinusne teoreme dobijamo:\\
$|DC|^2=|MD|^2+|MC|^2-2\cdot|MD|\cdot|MC|\cos(\gamma)$.
    Odavde je $\gamma=\arccos(\frac{\frac{1}{2}}{2\cdot\frac{3}{4}})=\arccos(\frac{1}{3})$.
    \par
    \subsection{Rastojanja}
    Rastojanje između tačke i prave (sa izvodjenjem),
    rastojanje između tačke i ravni (dokaz), rastojanje između
    mimoilaznih pravih (pitanje 16).
    \\
    \textbf{Odgovor:}\\
    Rastojanje između tačke i prave:\\
    Neka je data prava $p$ i tačka $A$ čije rastojanje se traži.
    Možemo da izaberemo neku tačku $P$ na pravi i da izračunamo
    površinu paralelograma zadatog vektorima: $\overrightarrow{AB}$ i $\overrightarrow{p}$.
    Tada se tražena visina dobija formulom: $d(A,p)=\frac{|\overrightarrow{AB}\times\overrightarrow{p}|}{|\overrightarrow{p}|}$.\\
Rastojanje između tačke i ravni: potpuno analogno osim što
se umesto površine i dužine koristi zapreina i površina.\\
\hyperlink{subsec:pitanje_16}{pitanje 16}.
\end{document}